\hypertarget{stm32f0xx__hal__i2c_8c}{}\section{H\+A\+L\+\_\+\+Driver/\+Src/stm32f0xx\+\_\+hal\+\_\+i2c.c File Reference}
\label{stm32f0xx__hal__i2c_8c}\index{H\+A\+L\+\_\+\+Driver/\+Src/stm32f0xx\+\_\+hal\+\_\+i2c.\+c@{H\+A\+L\+\_\+\+Driver/\+Src/stm32f0xx\+\_\+hal\+\_\+i2c.\+c}}


I2C H\+AL module driver. This file provides firmware functions to manage the following functionalities of the Inter Integrated Circuit (I2C) peripheral\+:  


{\ttfamily \#include \char`\"{}stm32f0xx\+\_\+hal.\+h\char`\"{}}\newline


\subsection{Detailed Description}
I2C H\+AL module driver. This file provides firmware functions to manage the following functionalities of the Inter Integrated Circuit (I2C) peripheral\+: 

\begin{DoxyAuthor}{Author}
M\+CD Application Team 
\end{DoxyAuthor}
\begin{DoxyVersion}{Version}
V1.\+5.\+0 
\end{DoxyVersion}
\begin{DoxyDate}{Date}
04-\/\+November-\/2016
\begin{DoxyItemize}
\item Initialization and de-\/initialization functions
\item IO operation functions
\item Peripheral State and Errors functions
\end{DoxyItemize}
\end{DoxyDate}
\begin{DoxyVerb}==============================================================================
                      ##### How to use this driver #####
==============================================================================
  [..]
  The I2C HAL driver can be used as follows:

  (#) Declare a I2C_HandleTypeDef handle structure, for example:
      I2C_HandleTypeDef  hi2c;

  (#)Initialize the I2C low level resources by implementing the HAL_I2C_MspInit() API:
      (##) Enable the I2Cx interface clock
      (##) I2C pins configuration
          (+++) Enable the clock for the I2C GPIOs
          (+++) Configure I2C pins as alternate function open-drain
      (##) NVIC configuration if you need to use interrupt process
          (+++) Configure the I2Cx interrupt priority
          (+++) Enable the NVIC I2C IRQ Channel
      (##) DMA Configuration if you need to use DMA process
          (+++) Declare a DMA_HandleTypeDef handle structure for the transmit or receive channel
          (+++) Enable the DMAx interface clock using
          (+++) Configure the DMA handle parameters
          (+++) Configure the DMA Tx or Rx channel
          (+++) Associate the initialized DMA handle to the hi2c DMA Tx or Rx handle
          (+++) Configure the priority and enable the NVIC for the transfer complete interrupt on
                the DMA Tx or Rx channel

  (#) Configure the Communication Clock Timing, Own Address1, Master Addressing mode, Dual Addressing mode,
      Own Address2, Own Address2 Mask, General call and Nostretch mode in the hi2c Init structure.

  (#) Initialize the I2C registers by calling the HAL_I2C_Init(), configures also the low level Hardware
      (GPIO, CLOCK, NVIC...etc) by calling the customized HAL_I2C_MspInit(&hi2c) API.

  (#) To check if target device is ready for communication, use the function HAL_I2C_IsDeviceReady()

  (#) For I2C IO and IO MEM operations, three operation modes are available within this driver :

  *** Polling mode IO operation ***
  =================================
  [..]
    (+) Transmit in master mode an amount of data in blocking mode using HAL_I2C_Master_Transmit()
    (+) Receive in master mode an amount of data in blocking mode using HAL_I2C_Master_Receive()
    (+) Transmit in slave mode an amount of data in blocking mode using HAL_I2C_Slave_Transmit()
    (+) Receive in slave mode an amount of data in blocking mode using HAL_I2C_Slave_Receive()

  *** Polling mode IO MEM operation ***
  =====================================
  [..]
    (+) Write an amount of data in blocking mode to a specific memory address using HAL_I2C_Mem_Write()
    (+) Read an amount of data in blocking mode from a specific memory address using HAL_I2C_Mem_Read()


  *** Interrupt mode IO operation ***
  ===================================
  [..]
    (+) Transmit in master mode an amount of data in non-blocking mode using HAL_I2C_Master_Transmit_IT()
    (+) At transmission end of transfer, HAL_I2C_MasterTxCpltCallback() is executed and user can
         add his own code by customization of function pointer HAL_I2C_MasterTxCpltCallback()
    (+) Receive in master mode an amount of data in non-blocking mode using HAL_I2C_Master_Receive_IT()
    (+) At reception end of transfer, HAL_I2C_MasterRxCpltCallback() is executed and user can
         add his own code by customization of function pointer HAL_I2C_MasterRxCpltCallback()
    (+) Transmit in slave mode an amount of data in non-blocking mode using HAL_I2C_Slave_Transmit_IT()
    (+) At transmission end of transfer, HAL_I2C_SlaveTxCpltCallback() is executed and user can
         add his own code by customization of function pointer HAL_I2C_SlaveTxCpltCallback()
    (+) Receive in slave mode an amount of data in non-blocking mode using HAL_I2C_Slave_Receive_IT()
    (+) At reception end of transfer, HAL_I2C_SlaveRxCpltCallback() is executed and user can
         add his own code by customization of function pointer HAL_I2C_SlaveRxCpltCallback()
    (+) In case of transfer Error, HAL_I2C_ErrorCallback() function is executed and user can
         add his own code by customization of function pointer HAL_I2C_ErrorCallback()
    (+) Abort a master I2C process communication with Interrupt using HAL_I2C_Master_Abort_IT()
    (+) End of abort process, HAL_I2C_AbortCpltCallback() is executed and user can
         add his own code by customization of function pointer HAL_I2C_AbortCpltCallback()
    (+) Discard a slave I2C process communication using __HAL_I2C_GENERATE_NACK() macro.
         This action will inform Master to generate a Stop condition to discard the communication.


  *** Interrupt mode IO sequential operation ***
  ==============================================
  [..]
    (@) These interfaces allow to manage a sequential transfer with a repeated start condition
        when a direction change during transfer
  [..]
    (+) A specific option field manage the different steps of a sequential transfer
    (+) Option field values are defined through @ref I2C_XFEROPTIONS and are listed below:
    (++) I2C_FIRST_AND_LAST_FRAME: No sequential usage, functionnal is same as associated interfaces in no sequential mode
    (++) I2C_FIRST_FRAME: Sequential usage, this option allow to manage a sequence with start condition, address
                          and data to transfer without a final stop condition
    (++) I2C_FIRST_AND_NEXT_FRAME: Sequential usage (Master only), this option allow to manage a sequence with start condition, address
                          and data to transfer without a final stop condition, an then permit a call the same master sequential interface
                          several times (like HAL_I2C_Master_Sequential_Transmit_IT() then HAL_I2C_Master_Sequential_Transmit_IT())
    (++) I2C_NEXT_FRAME: Sequential usage, this option allow to manage a sequence with a restart condition, address
                          and with new data to transfer if the direction change or manage only the new data to transfer
                          if no direction change and without a final stop condition in both cases
    (++) I2C_LAST_FRAME: Sequential usage, this option allow to manage a sequance with a restart condition, address
                          and with new data to transfer if the direction change or manage only the new data to transfer
                          if no direction change and with a final stop condition in both cases

    (+) Differents sequential I2C interfaces are listed below:
    (++) Sequential transmit in master I2C mode an amount of data in non-blocking mode using HAL_I2C_Master_Sequential_Transmit_IT()
    (+++) At transmission end of current frame transfer, HAL_I2C_MasterTxCpltCallback() is executed and user can
         add his own code by customization of function pointer HAL_I2C_MasterTxCpltCallback()
    (++) Sequential receive in master I2C mode an amount of data in non-blocking mode using HAL_I2C_Master_Sequential_Receive_IT()
    (+++) At reception end of current frame transfer, HAL_I2C_MasterRxCpltCallback() is executed and user can
         add his own code by customization of function pointer HAL_I2C_MasterRxCpltCallback()
    (++) Abort a master I2C process communication with Interrupt using HAL_I2C_Master_Abort_IT()
    (+++) End of abort process, HAL_I2C_AbortCpltCallback() is executed and user can
         add his own code by customization of function pointer HAL_I2C_AbortCpltCallback()
    (++) Enable/disable the Address listen mode in slave I2C mode using HAL_I2C_EnableListen_IT() HAL_I2C_DisableListen_IT()
    (+++) When address slave I2C match, HAL_I2C_AddrCallback() is executed and user can
         add his own code to check the Address Match Code and the transmission direction request by master (Write/Read).
    (+++) At Listen mode end HAL_I2C_ListenCpltCallback() is executed and user can
         add his own code by customization of function pointer HAL_I2C_ListenCpltCallback()
    (++) Sequential transmit in slave I2C mode an amount of data in non-blocking mode using HAL_I2C_Slave_Sequential_Transmit_IT()
    (+++) At transmission end of current frame transfer, HAL_I2C_SlaveTxCpltCallback() is executed and user can
         add his own code by customization of function pointer HAL_I2C_SlaveTxCpltCallback()
    (++) Sequential receive in slave I2C mode an amount of data in non-blocking mode using HAL_I2C_Slave_Sequential_Receive_IT()
    (+++) At reception end of current frame transfer, HAL_I2C_SlaveRxCpltCallback() is executed and user can
         add his own code by customization of function pointer HAL_I2C_SlaveRxCpltCallback()
    (++) In case of transfer Error, HAL_I2C_ErrorCallback() function is executed and user can
         add his own code by customization of function pointer HAL_I2C_ErrorCallback()
    (++) Abort a master I2C process communication with Interrupt using HAL_I2C_Master_Abort_IT()
    (++) End of abort process, HAL_I2C_AbortCpltCallback() is executed and user can
         add his own code by customization of function pointer HAL_I2C_AbortCpltCallback()
    (++) Discard a slave I2C process communication using __HAL_I2C_GENERATE_NACK() macro.
         This action will inform Master to generate a Stop condition to discard the communication.

  *** Interrupt mode IO MEM operation ***
  =======================================
  [..]
    (+) Write an amount of data in non-blocking mode with Interrupt to a specific memory address using
        HAL_I2C_Mem_Write_IT()
    (+) At Memory end of write transfer, HAL_I2C_MemTxCpltCallback() is executed and user can
         add his own code by customization of function pointer HAL_I2C_MemTxCpltCallback()
    (+) Read an amount of data in non-blocking mode with Interrupt from a specific memory address using
        HAL_I2C_Mem_Read_IT()
    (+) At Memory end of read transfer, HAL_I2C_MemRxCpltCallback() is executed and user can
         add his own code by customization of function pointer HAL_I2C_MemRxCpltCallback()
    (+) In case of transfer Error, HAL_I2C_ErrorCallback() function is executed and user can
         add his own code by customization of function pointer HAL_I2C_ErrorCallback()

  *** DMA mode IO operation ***
  ==============================
  [..]
    (+) Transmit in master mode an amount of data in non-blocking mode (DMA) using
        HAL_I2C_Master_Transmit_DMA()
    (+) At transmission end of transfer, HAL_I2C_MasterTxCpltCallback() is executed and user can
         add his own code by customization of function pointer HAL_I2C_MasterTxCpltCallback()
    (+) Receive in master mode an amount of data in non-blocking mode (DMA) using
        HAL_I2C_Master_Receive_DMA()
    (+) At reception end of transfer, HAL_I2C_MasterRxCpltCallback() is executed and user can
         add his own code by customization of function pointer HAL_I2C_MasterRxCpltCallback()
    (+) Transmit in slave mode an amount of data in non-blocking mode (DMA) using
        HAL_I2C_Slave_Transmit_DMA()
    (+) At transmission end of transfer, HAL_I2C_SlaveTxCpltCallback() is executed and user can
         add his own code by customization of function pointer HAL_I2C_SlaveTxCpltCallback()
    (+) Receive in slave mode an amount of data in non-blocking mode (DMA) using
        HAL_I2C_Slave_Receive_DMA()
    (+) At reception end of transfer, HAL_I2C_SlaveRxCpltCallback() is executed and user can
         add his own code by customization of function pointer HAL_I2C_SlaveRxCpltCallback()
    (+) In case of transfer Error, HAL_I2C_ErrorCallback() function is executed and user can
         add his own code by customization of function pointer HAL_I2C_ErrorCallback()
    (+) Abort a master I2C process communication with Interrupt using HAL_I2C_Master_Abort_IT()
    (+) End of abort process, HAL_I2C_AbortCpltCallback() is executed and user can
         add his own code by customization of function pointer HAL_I2C_AbortCpltCallback()
    (+) Discard a slave I2C process communication using __HAL_I2C_GENERATE_NACK() macro.
         This action will inform Master to generate a Stop condition to discard the communication.

  *** DMA mode IO MEM operation ***
  =================================
  [..]
    (+) Write an amount of data in non-blocking mode with DMA to a specific memory address using
        HAL_I2C_Mem_Write_DMA()
    (+) At Memory end of write transfer, HAL_I2C_MemTxCpltCallback() is executed and user can
         add his own code by customization of function pointer HAL_I2C_MemTxCpltCallback()
    (+) Read an amount of data in non-blocking mode with DMA from a specific memory address using
        HAL_I2C_Mem_Read_DMA()
    (+) At Memory end of read transfer, HAL_I2C_MemRxCpltCallback() is executed and user can
         add his own code by customization of function pointer HAL_I2C_MemRxCpltCallback()
    (+) In case of transfer Error, HAL_I2C_ErrorCallback() function is executed and user can
         add his own code by customization of function pointer HAL_I2C_ErrorCallback()


   *** I2C HAL driver macros list ***
   ==================================
   [..]
     Below the list of most used macros in I2C HAL driver.

    (+) __HAL_I2C_ENABLE: Enable the I2C peripheral
    (+) __HAL_I2C_DISABLE: Disable the I2C peripheral
    (+) __HAL_I2C_GENERATE_NACK: Generate a Non-Acknowledge I2C peripheral in Slave mode
    (+) __HAL_I2C_GET_FLAG: Check whether the specified I2C flag is set or not
    (+) __HAL_I2C_CLEAR_FLAG: Clear the specified I2C pending flag
    (+) __HAL_I2C_ENABLE_IT: Enable the specified I2C interrupt
    (+) __HAL_I2C_DISABLE_IT: Disable the specified I2C interrupt

   [..]
     (@) You can refer to the I2C HAL driver header file for more useful macros\end{DoxyVerb}


\begin{DoxyAttention}{Attention}

\end{DoxyAttention}
\subsubsection*{\begin{center}\copyright{} C\+O\+P\+Y\+R\+I\+G\+H\+T(c) 2016 S\+T\+Microelectronics\end{center} }

Redistribution and use in source and binary forms, with or without modification, are permitted provided that the following conditions are met\+:
\begin{DoxyEnumerate}
\item Redistributions of source code must retain the above copyright notice, this list of conditions and the following disclaimer.
\item Redistributions in binary form must reproduce the above copyright notice, this list of conditions and the following disclaimer in the documentation and/or other materials provided with the distribution.
\item Neither the name of S\+T\+Microelectronics nor the names of its contributors may be used to endorse or promote products derived from this software without specific prior written permission.
\end{DoxyEnumerate}

T\+H\+IS S\+O\+F\+T\+W\+A\+RE IS P\+R\+O\+V\+I\+D\+ED BY T\+HE C\+O\+P\+Y\+R\+I\+G\+HT H\+O\+L\+D\+E\+RS A\+ND C\+O\+N\+T\+R\+I\+B\+U\+T\+O\+RS \char`\"{}\+A\+S I\+S\char`\"{} A\+ND A\+NY E\+X\+P\+R\+E\+SS OR I\+M\+P\+L\+I\+ED W\+A\+R\+R\+A\+N\+T\+I\+ES, I\+N\+C\+L\+U\+D\+I\+NG, B\+UT N\+OT L\+I\+M\+I\+T\+ED TO, T\+HE I\+M\+P\+L\+I\+ED W\+A\+R\+R\+A\+N\+T\+I\+ES OF M\+E\+R\+C\+H\+A\+N\+T\+A\+B\+I\+L\+I\+TY A\+ND F\+I\+T\+N\+E\+SS F\+OR A P\+A\+R\+T\+I\+C\+U\+L\+AR P\+U\+R\+P\+O\+SE A\+RE D\+I\+S\+C\+L\+A\+I\+M\+ED. IN NO E\+V\+E\+NT S\+H\+A\+LL T\+HE C\+O\+P\+Y\+R\+I\+G\+HT H\+O\+L\+D\+ER OR C\+O\+N\+T\+R\+I\+B\+U\+T\+O\+RS BE L\+I\+A\+B\+LE F\+OR A\+NY D\+I\+R\+E\+CT, I\+N\+D\+I\+R\+E\+CT, I\+N\+C\+I\+D\+E\+N\+T\+AL, S\+P\+E\+C\+I\+AL, E\+X\+E\+M\+P\+L\+A\+RY, OR C\+O\+N\+S\+E\+Q\+U\+E\+N\+T\+I\+AL D\+A\+M\+A\+G\+ES (I\+N\+C\+L\+U\+D\+I\+NG, B\+UT N\+OT L\+I\+M\+I\+T\+ED TO, P\+R\+O\+C\+U\+R\+E\+M\+E\+NT OF S\+U\+B\+S\+T\+I\+T\+U\+TE G\+O\+O\+DS OR S\+E\+R\+V\+I\+C\+ES; L\+O\+SS OF U\+SE, D\+A\+TA, OR P\+R\+O\+F\+I\+TS; OR B\+U\+S\+I\+N\+E\+SS I\+N\+T\+E\+R\+R\+U\+P\+T\+I\+ON) H\+O\+W\+E\+V\+ER C\+A\+U\+S\+ED A\+ND ON A\+NY T\+H\+E\+O\+RY OF L\+I\+A\+B\+I\+L\+I\+TY, W\+H\+E\+T\+H\+ER IN C\+O\+N\+T\+R\+A\+CT, S\+T\+R\+I\+CT L\+I\+A\+B\+I\+L\+I\+TY, OR T\+O\+RT (I\+N\+C\+L\+U\+D\+I\+NG N\+E\+G\+L\+I\+G\+E\+N\+CE OR O\+T\+H\+E\+R\+W\+I\+SE) A\+R\+I\+S\+I\+NG IN A\+NY W\+AY O\+UT OF T\+HE U\+SE OF T\+H\+IS S\+O\+F\+T\+W\+A\+RE, E\+V\+EN IF A\+D\+V\+I\+S\+ED OF T\+HE P\+O\+S\+S\+I\+B\+I\+L\+I\+TY OF S\+U\+CH D\+A\+M\+A\+GE. 